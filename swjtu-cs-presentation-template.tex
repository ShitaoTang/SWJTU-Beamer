\documentclass[12pt, aspectratio=169]{beamer}

% SWJTU CS theme
\usetheme{swjtucs}
% Packages for additional features
\usepackage{graphicx}
\usepackage{booktabs}
\usepackage{amsmath}
\usepackage{xcolor}
\usepackage{fontspec}
\usepackage{unicode-math}
\usepackage{subcaption}
\usepackage{caption}

\usepackage{code-style}

\definecolor{red}{RGB}{220,20,60}  % A vibrant red
\definecolor{blue}{RGB}{30,144,255} % A bright blue

% Ensure numbered captions in Beamer
\setbeamertemplate{caption}[numbered]

% Caption setup for numbered listings
\DeclareCaptionType{codelisting}[Listing]{}

% For hyperlinks
\usepackage{hyperref}
\hypersetup{
    colorlinks=true,
    linkcolor=blue,
    urlcolor=blue,
    citecolor=blue
}

\title{Title goes here and it can be quite long that it will wrap onto a 2nd and 3rd line}
\author{Shitao Tang}
\institute{Southwest Jiaotong University, China}
\email{tst17@my.swjtu.edu.cn}

\begin{document}
    % Title page
    \begin{frame}
        \titlepage
    \end{frame}

    % Frame 1: Ordinary text, italics, bold, texttt, red/blue fonts, footnote
    \begin{frame}{Basic Text Formatting}
        Ordinary text is straightforward in Beamer.

        \emph{This is italic text} (using \texttt{\textbackslash emph} or \texttt{\textbackslash itshape}).

        \textbf{This is bold text} (using \texttt{\textbackslash textbf}).

        \texttt{This is monospace text} (using \texttt{\textbackslash texttt}).

        \textcolor{red}{This text is in red} and \textcolor{blue}{this is in blue}.

        A footnote example\footnote{This is a sample footnote in Beamer.}.
    \end{frame}

    % Frame 2: Bullet points (itemize/enumerate)
    \begin{frame}{Bullet Points and Lists}
        \begin{itemize}
            \item First bullet point: Basic itemize environment.
            \item Second bullet point: Supports nesting.
                \begin{itemize}
                    \item Nested item.
                \end{itemize}
            \item Third: With emphasis \emph{here}.
        \end{itemize}

        For numbered lists:
        \begin{enumerate}
            \item First step.
            \item Second step.
        \end{enumerate}
    \end{frame}

    % Frame 3: Math formulas (numbered and unnumbered)
    \begin{frame}{Mathematical Formulas}
        Unnumbered inline math: $E = mc^2$.

        Display math without numbering:
        \begin{equation*}
            \int_{-\infty}^{\infty} e^{-x^2} \, dx = \sqrt{\pi}
        \end{equation*}

        Mathematical constants and special functions:
        \begin{equation*}
            \uppi = 3.14\dots; \quad
            \symup{i}^2 = -1; \quad
            \symup{e} = \lim_{n \to \infty} \left( 1 + \frac{1}{n} \right)^n.
        \end{equation*}

        Numbered equation:
        \begin{equation}
            \nabla \cdot \mathbf{E} = \frac{\rho}{\epsilon_0}
            \label{eq:gauss}
        \end{equation}

        Reference: See Equation~\ref{eq:gauss}.
    \end{frame}

    % Frame 4: Single image
    \begin{frame}{Single Image}
        \begin{figure}
            \centering
            \includegraphics[width=0.5\textwidth]{example-image-a} % Replace with your image path
            \caption{A sample image centered on the slide.}
            \label{fig:sample}
        \end{figure}
    \end{frame}

    % Frame 5: Two images side-by-side
    \begin{frame}{Two Images Side-by-Side}
        \begin{figure}
            \centering
            \begin{subfigure}{0.48\textwidth}
                \centering
                \includegraphics[width=\textwidth]{example-image-b}
                \caption{First image.}
                \label{fig:two-a}
            \end{subfigure}
            \hfill
            \begin{subfigure}{0.48\textwidth}
                \centering
                \includegraphics[width=\textwidth]{example-image-c}
                \caption{Second image.}
                \label{fig:two-b}
            \end{subfigure}
            \caption{Two images side-by-side.}
            \label{fig:two}
        \end{figure}
    \end{frame}

    % Frame 6: Three images side-by-side
    \begin{frame}{Three Images Side-by-Side}
        \begin{figure}
            \centering
            \begin{subfigure}{0.31\textwidth}
                \centering
                \includegraphics[width=\textwidth]{example-image}
                \caption{First image.}
                \label{fig:three-a}
            \end{subfigure}
            \hfill
            \begin{subfigure}{0.31\textwidth}
                \centering
                \includegraphics[width=\textwidth]{example-image-a}
                \caption{Second image.}
                \label{fig:three-b}
            \end{subfigure}
            \hfill
            \begin{subfigure}{0.31\textwidth}
                \centering
                \includegraphics[width=\textwidth]{example-image-b}
                \caption{Third image.}
                \label{fig:three-c}
            \end{subfigure}
            \caption{Three images side-by-side.}
            \label{fig:three}
        \end{figure}
    \end{frame}

    % Frame 7: Left image, right text
    \begin{frame}{Image on Left, Text on Right}
        \begin{columns}[T]
            \begin{column}{0.4\textwidth}
                \begin{figure}
                    \centering
                    \includegraphics[width=\textwidth]{example-image}
                    \caption{Sample image.}
                    \label{fig:left}
                \end{figure}
            \end{column}
            \begin{column}{0.6\textwidth}
                This is text on the right side, aligned with the image on the left.
                It can span multiple lines and include lists:
                \begin{itemize}
                    \item Point one.
                    \item Point two.
                \end{itemize}
            \end{column}
        \end{columns}
    \end{frame}

    % Frame 8: Python code segment
    \begin{frame}[fragile]{Python Code Segment}
        \begin{codelisting}
            \begin{lstlisting}[language=Python]
def hello_world(name):
    """Greet someone."""
    print(f"Hello, {name}!")
    
    if name == "World":
        return "Global greeting"
    return "Personal greeting"

hello_world("SWJTU")
            \end{lstlisting}
            \caption{Hello World in Python}
            \label{lst:python}
        \end{codelisting}
    \end{frame}

    % Frame 9: C code segment
    \begin{frame}[fragile]{C Code Segment}
        \begin{codelisting}
            \begin{lstlisting}[language=C]
#include <stdio.h>

int main() {
    printf("Hello, World!\n");
    return 0;
}
            \end{lstlisting}
            \caption{Hello World in C}
            \label{lst:c}
        \end{codelisting}
    \end{frame}

    % Frame 10: Table
    \begin{frame}{Tables}
        A simple table using booktabs:
        \begin{table}
            \centering
            \begin{tabular}{lcc}
                \toprule
                Item & Price & Quantity \\
                \midrule
                Widget A & \$10 & 5 \\
                Widget B & \$15 & 3 \\
                \bottomrule
            \end{tabular}
            \caption{Sample inventory table.}
            \label{tab:inventory}
        \end{table}

        Reference: See Table~\ref{tab:inventory}.
    \end{frame}

    % Frame 11: Hyperlinks and references
    \begin{frame}{Hyperlinks and References}
        \begin{itemize}
            \item Link to a URL: \href{https://www.latex-project.org/}{LaTeX Project}.
            \item Internal references:
                \begin{itemize}
                    \item Equation~\ref{eq:gauss}
                    \item Figure~\ref{fig:sample}
                    \item Figure~\ref{fig:two} (subfigures \ref{fig:two-a} and \ref{fig:two-b})
                    \item Figure~\ref{fig:three} (subfigures \ref{fig:three-a}, \ref{fig:three-b}, \ref{fig:three-c})
                    \item Figure~\ref{fig:left}
                    \item Table~\ref{tab:inventory}
                    \item Listing~\ref{lst:python}
                    \item Listing~\ref{lst:c}
                \end{itemize}
            \item Alert: \alert{This is highlighted text} for emphasis.
        \end{itemize}

        Block example:
        \begin{block}{Key Insight}
            Beamer themes like yours can be customized extensively.
        \end{block}
    \end{frame}

    % Frame 12: Additional elements (blocks, theorems, etc.)
    \begin{frame}{Additional Features}
        Theorem-like environment (requires amsthm, but using Beamer's theorem):
        \begin{theorem}
            Pythagorean theorem: $a^2 + b^2 = c^2$.
        \end{theorem}

        \vspace{1em}

        Example block:
        \begin{example}
            For $x = 3$, $f(x) = x^2 = 9$.
        \end{example}

        \vspace{1em}

        Alert block:
        \begin{alertblock}{Warning}
            Always validate inputs!
        \end{alertblock}
    \end{frame}

\end{document}
